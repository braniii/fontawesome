\documentclass{scrartcl}
\usepackage{hyperref}
\usepackage{shortvrb}
\usepackage{metalogo}
\usepackage{longtable}
\usepackage{xparse}
\usepackage[fa5alias]{fontawesome6}
%\usepackage{fontawesome6}
% \usepackage{xcolor}
% \usepackage[pro]{fontawesome6}
% \faStyle{duotone-solid}
\usepackage[utf8]{inputenc}
\usepackage{geometry}
\MakeShortVerb{\|}
\usepackage[english]{babel}
\begin{document}
\title{The \texttt{fontawesome6} Package\thanks{This document corresponds to fontawesome6 version 6.7.2-2, dated 2025/05/03.}}
\author{Font Awesome\thanks{More information at \url{https://fontawesome.com}} (The font)\and Daniel Nagel\thanks{GitHub: \href{https://github.com/braniii}{github.com/braniii}} (The \LaTeX{} package)}
\maketitle

This package provides \LaTeX{} support for Font Awesome 6 icons.

Special thanks to Marcel Krüger for the original \texttt{fontawesome5} package\\
(\url{https://ctan.org/pkg/fontawesome5}), upon which this package is based.

\subsection*{Usage}

To use Font Awesome 6 icons in your document, load the package with:
\begin{verbatim}
  \usepackage{fontawesome6}
\end{verbatim}
Optionally, you can add the |fixed| option to enable fixed-width icons:
\begin{verbatim}
  \usepackage[fixed]{fontawesome6}
\end{verbatim}

Each icon is available as a macro, using the official icon name in CamelCase with the prefix |\fa|.\\
For example, to use the |hand-point-up| icon, write |\faHandPointUp|.\\
An optional argument allows you to select the style (|solid| or |regular|). The
default style is |solid|, but you can change it globally with |\faStyle{...}| or
by setting the option |style=...|.

Alternatively, you can access any icon by its official name using |\faIcon{icon-name}| or |\faIcon[style]{icon-name}|.

A comprehensive list of all included icons and their corresponding commands is provided at the end of this document.

\subsection*{Example}
\begin{verbatim}
...
\usepackage{fontawesome6}
...
\begin{document}
...
A simple icon: \faHandPointUp\\
Multiple versions of the file icon:
  \faFile~
  \faFile[solid]~
  \faFile[regular]~.\\
Alternative syntax:
  \faIcon{file}~
  \faIcon[solid]{file}~
  \faIcon[regular]{file}~.
...
\end{document}
\end{verbatim}

A simple icon: \faHandPointUp\\
Multiple versions of the file icon: \faFile~\faFile[solid]~\faFile[regular].\\
Alternative syntax: \faIcon{file}~\faIcon[solid]{file}~\faIcon[regular]{file}.

\subsection*{Font Awesome Pro}
Font Awesome 6 is available in both Free and Pro versions. By default, this package uses the Free version. If you have a Pro license and have installed the Font Awesome 6 Pro desktop fonts in your system font path, you can enable Pro support by loading the package with the |[pro]| option:
\begin{verbatim}
  \usepackage[pro]{fontawesome6}
\end{verbatim}

With Pro enabled, the following additional styles are available: |solid|, |regular|, |light|, |thin|, |duotone-solid|, |duotone-regular|, |duotone-light|, |duotone-thin|, |sharp-solid|, |sharp-regular|, |sharp-light|, |sharp-thin|, |sharp-duotone-solid|, |sharp-duotone-regular|, |sharp-duotone-light|, and |sharp-duotone-thin|.

For duotone icons, you can set the secondary color using |\faDuotoneSetSecondary|:
\begin{verbatim}
  % Remember to load xcolor
  % Set secondary color to green
  \faDuotoneSetSecondary{\color{green}}
\end{verbatim}
Pro features are supported only with \XeLaTeX{} and \LuaLaTeX.

\subsection*{Updates}
This package corresponds to Font Awesome 6.7.2.\\
If a newer version is available on the Font Awesome website, check for updates at \url{https://ctan.org/pkg/fontawesome6}. If the latest version is not yet on CTAN, you may contact \href{mailto:tex@2krueger.de}{\nolinkurl{tex@2krueger.de}}.

If you use \XeLaTeX{} or \LuaLaTeX{}, you can also manually download the new Desktop Fonts from \url{https://fontawesome.com} and place them in your \TeX{} tree. Save them with the following filenames:
{\ttfamily
\begin{tabular}{l}
  FontAwesome6Brands-Regular-400.otf\\
  FontAwesome6Free-Regular-400.otf\\
  FontAwesome6Free-Solid-900.otf
\end{tabular}
}\\
The package will then automatically use the new version.

\subsection*{Bugs and Feedback}
For bug reports or feature requests, please open an issue at \href{https://github.com/braniii/fontawesome}{github.com/braniii/fontawesome}.

\ExplSyntaxOn
\msg_new:nnnn {fontawesome6} {list/no-shorthand} {No~shorthand~defined~for~icon~#1.} {
  It~looks~like~#1~need~special~handling~in~fulllist.tex~but~there~are~
  no~appropriate~definitions.~Ask~a~wizard~to~add~#1~to~fulllist.tex~to~
  fix~this.
}
\tl_new:N \g__fontawesome_last_name_tl
\tl_new:N \g__fontawesome_last_cs_tl
\prg_new_protected_conditional:Nnn \__fontawesome_if_regular_style:nn {T} {
  \group_begin:
    \usefont{U}{fontawesome#1}{regular}{n}
    \iffontchar\font#2
      \group_insert_after:N \prg_return_true:
    \else:
      \group_insert_after:N \prg_return_false:
    \fi:
  \group_end:
}
\tracingonline1
\showboxdepth\maxdimen
\showboxbreadth\maxdimen
\cs_new:Nn\__fontawesome_list_show_icon:nnnn{
  \str_if_in:nnT{#3}{brands}{
    \hfilneg\vbox to0.875em{\vfil\hbox to0pt{\hss\tiny\faTrademark\quad}\vfil}\hfil
  }
  \faIcon{#2}&\texttt{\textbackslash#1}&\texttt{\textbackslash faIcon\{#2\}}
  \str_if_in:nnT{#3}{free}{
    \__fontawesome_if_regular_style:nnT {#3} {#4} {
      \\\faIcon[regular]{#2}&\texttt{\textbackslash#1[regular]}&\texttt{\textbackslash faIcon[regular]\{#2\}}
    }
  }
  \tl_gset:Nn \g__fontawesome_last_cs_tl {#1}
  \tl_gset:Nn \g__fontawesome_last_name_tl {#2}
  \\
}
\cs_generate_variant:Nn \__fontawesome_list_show_icon:nnnn { fnnn }
\cs_set:Nn\__fontawesome_def_icon:nnnnn{
  \__fontawesome_list_show_icon:fnnn{
    \tl_if_empty:nTF{#1}{
      faIcon\{\str_foldcase:n{#2}\}
    }{
      \cs_to_str:N #1 % You might have noticed that #1 is a n-type argument, not N-type.
      % This is not a mistake, the argument might contain additional characters after the initial cs
      % which is passed to \cs_to_str:N
    }
  }{#2}{#3}{#4}
}
\ExplSyntaxOff
\newgeometry{textwidth=18cm}
\subsection*{Full~icon~list~for~FontAwesome~6~Free}
All icons marked with \vbox to0.875em{\vfil\hbox{\hss\tiny\faTrademark}\vfil} are brand icons.
\begin{quote}
  All brand icons are trademarks of their respective owners. The use of these
  trademarks does not indicate endorsement of the trademark holder by Font
  Awesome, nor vice versa. \emph{Please do not use brand logos for any purpose except
  to represent the company, product, or service to which they refer.}
\end{quote}
\ExplSyntaxOn
\begin{longtable}{cll}
  \cs:w @@input\cs_end: fontawesome6-mapping.def~
\end{longtable}
\ExplSyntaxOff
\restoregeometry

% Auto-generated FA5 alias table

ewgeometry{textwidth=18cm}
\subsection*{Full~icon~list~for~FontAwesome~5~Aliases}
\begin{longtable}{cll}
\faAd & \texttt{\textbackslash faAd} & \texttt{\textbackslash faIcon{rectangle-ad}} \\
\faAdjust & \texttt{\textbackslash faAdjust} & \texttt{\textbackslash faIcon{circle-half-stroke}} \\
\faAirFreshener & \texttt{\textbackslash faAirFreshener} & \texttt{\textbackslash faIcon{spray-can-sparkles}} \\
\faAllergies & \texttt{\textbackslash faAllergies} & \texttt{\textbackslash faIcon{hand-dots}} \\
\faAmbulance & \texttt{\textbackslash faAmbulance} & \texttt{\textbackslash faIcon{truck-medical}} \\
\faAmericanSignLanguageInterpreting & \texttt{\textbackslash faAmericanSignLanguageInterpreting} & \texttt{\textbackslash faIcon{hands-asl-interpreting}} \\
\faAngleDoubleDown & \texttt{\textbackslash faAngleDoubleDown} & \texttt{\textbackslash faIcon{angles-down}} \\
\faAngleDoubleLeft & \texttt{\textbackslash faAngleDoubleLeft} & \texttt{\textbackslash faIcon{angles-left}} \\
\faAngleDoubleRight & \texttt{\textbackslash faAngleDoubleRight} & \texttt{\textbackslash faIcon{angles-right}} \\
\faAngleDoubleUp & \texttt{\textbackslash faAngleDoubleUp} & \texttt{\textbackslash faIcon{angles-up}} \\
\faAngry & \texttt{\textbackslash faAngry} & \texttt{\textbackslash faIcon{face-angry}} \\
\faAppleAlt & \texttt{\textbackslash faAppleAlt} & \texttt{\textbackslash faIcon{apple-whole}} \\
\faArchive & \texttt{\textbackslash faArchive} & \texttt{\textbackslash faIcon{box-archive}} \\
\faArrowAltCircleDown & \texttt{\textbackslash faArrowAltCircleDown} & \texttt{\textbackslash faIcon{circle-down}} \\
\faArrowAltCircleLeft & \texttt{\textbackslash faArrowAltCircleLeft} & \texttt{\textbackslash faIcon{circle-left}} \\
\faArrowAltCircleRight & \texttt{\textbackslash faArrowAltCircleRight} & \texttt{\textbackslash faIcon{circle-right}} \\
\faArrowAltCircleUp & \texttt{\textbackslash faArrowAltCircleUp} & \texttt{\textbackslash faIcon{circle-up}} \\
\faArrowAltDown & \texttt{\textbackslash faArrowAltDown} & \texttt{\textbackslash faIcon{arrow-down}} \\
\faArrowAltLeft & \texttt{\textbackslash faArrowAltLeft} & \texttt{\textbackslash faIcon{arrow-left}} \\
\faArrowAltRight & \texttt{\textbackslash faArrowAltRight} & \texttt{\textbackslash faIcon{arrow-right}} \\
\faArrowAltUp & \texttt{\textbackslash faArrowAltUp} & \texttt{\textbackslash faIcon{arrow-up}} \\
\faArrowCircleDown & \texttt{\textbackslash faArrowCircleDown} & \texttt{\textbackslash faIcon{circle-arrow-down}} \\
\faArrowCircleLeft & \texttt{\textbackslash faArrowCircleLeft} & \texttt{\textbackslash faIcon{circle-arrow-left}} \\
\faArrowCircleRight & \texttt{\textbackslash faArrowCircleRight} & \texttt{\textbackslash faIcon{circle-arrow-right}} \\
\faArrowCircleUp & \texttt{\textbackslash faArrowCircleUp} & \texttt{\textbackslash faIcon{circle-arrow-up}} \\
\faAssistiveListeningSystems & \texttt{\textbackslash faAssistiveListeningSystems} & \texttt{\textbackslash faIcon{ear-listen}} \\
\faAtlas & \texttt{\textbackslash faAtlas} & \texttt{\textbackslash faIcon{book-atlas}} \\
\faAtomAlt & \texttt{\textbackslash faAtomAlt} & \texttt{\textbackslash faIcon{atom}} \\
\faBackspace & \texttt{\textbackslash faBackspace} & \texttt{\textbackslash faIcon{delete-left}} \\
\faBalanceScale & \texttt{\textbackslash faBalanceScale} & \texttt{\textbackslash faIcon{scale-balanced}} \\
\faBalanceScaleLeft & \texttt{\textbackslash faBalanceScaleLeft} & \texttt{\textbackslash faIcon{scale-unbalanced}} \\
\faBalanceScaleRight & \texttt{\textbackslash faBalanceScaleRight} & \texttt{\textbackslash faIcon{scale-unbalanced-flip}} \\
\faBandAid & \texttt{\textbackslash faBandAid} & \texttt{\textbackslash faIcon{bandage}} \\
\faBaseballBall & \texttt{\textbackslash faBaseballBall} & \texttt{\textbackslash faIcon{baseball}} \\
\faBasketballBall & \texttt{\textbackslash faBasketballBall} & \texttt{\textbackslash faIcon{basketball}} \\
\faBedAlt & \texttt{\textbackslash faBedAlt} & \texttt{\textbackslash faIcon{bed-front}} \\
\faBeer & \texttt{\textbackslash faBeer} & \texttt{\textbackslash faIcon{beer-mug-empty}} \\
\faBetamax & \texttt{\textbackslash faBetamax} & \texttt{\textbackslash faIcon{cassette-betamax}} \\
\faBible & \texttt{\textbackslash faBible} & \texttt{\textbackslash faIcon{book-bible}} \\
\faBiking & \texttt{\textbackslash faBiking} & \texttt{\textbackslash faIcon{person-biking}} \\
\faBikingMountain & \texttt{\textbackslash faBikingMountain} & \texttt{\textbackslash faIcon{person-biking-mountain}} \\
\faBirthdayCake & \texttt{\textbackslash faBirthdayCake} & \texttt{\textbackslash faIcon{cake-candles}} \\
\faBlind & \texttt{\textbackslash faBlind} & \texttt{\textbackslash faIcon{person-walking-with-cane}} \\
\faBookAlt & \texttt{\textbackslash faBookAlt} & \texttt{\textbackslash faIcon{book-blank}} \\
\faBookDead & \texttt{\textbackslash faBookDead} & \texttt{\textbackslash faIcon{book-skull}} \\
\faBookReader & \texttt{\textbackslash faBookReader} & \texttt{\textbackslash faIcon{book-open-reader}} \\
\faBookSpells & \texttt{\textbackslash faBookSpells} & \texttt{\textbackslash faIcon{book-sparkles}} \\
\faBorderStyle & \texttt{\textbackslash faBorderStyle} & \texttt{\textbackslash faIcon{border-top-left}} \\
\faBorderStyleAlt & \texttt{\textbackslash faBorderStyleAlt} & \texttt{\textbackslash faIcon{border-bottom-right}} \\
\faBoxAlt & \texttt{\textbackslash faBoxAlt} & \texttt{\textbackslash faIcon{box-taped}} \\
\faBoxFragile & \texttt{\textbackslash faBoxFragile} & \texttt{\textbackslash faIcon{square-fragile}} \\
\faBoxFull & \texttt{\textbackslash faBoxFull} & \texttt{\textbackslash faIcon{box-open-full}} \\
\faBoxUp & \texttt{\textbackslash faBoxUp} & \texttt{\textbackslash faIcon{square-this-way-up}} \\
\faBoxUsd & \texttt{\textbackslash faBoxUsd} & \texttt{\textbackslash faIcon{box-dollar}} \\
\faBoxes & \texttt{\textbackslash faBoxes} & \texttt{\textbackslash faIcon{boxes-stacked}} \\
\faBoxesAlt & \texttt{\textbackslash faBoxesAlt} & \texttt{\textbackslash faIcon{boxes-stacked}} \\
\faBrackets & \texttt{\textbackslash faBrackets} & \texttt{\textbackslash faIcon{brackets-square}} \\
\faBroadcastTower & \texttt{\textbackslash faBroadcastTower} & \texttt{\textbackslash faIcon{tower-broadcast}} \\
\faBurn & \texttt{\textbackslash faBurn} & \texttt{\textbackslash faIcon{fire-flame-simple}} \\
\faBusAlt & \texttt{\textbackslash faBusAlt} & \texttt{\textbackslash faIcon{bus-simple}} \\
\faCalculatorAlt & \texttt{\textbackslash faCalculatorAlt} & \texttt{\textbackslash faIcon{calculator-simple}} \\
\faCalendarAlt & \texttt{\textbackslash faCalendarAlt} & \texttt{\textbackslash faIcon{calendar-days}} \\
\faCalendarEdit & \texttt{\textbackslash faCalendarEdit} & \texttt{\textbackslash faIcon{calendar-pen}} \\
\faCalendarTimes & \texttt{\textbackslash faCalendarTimes} & \texttt{\textbackslash faIcon{calendar-xmark}} \\
\faCameraAlt & \texttt{\textbackslash faCameraAlt} & \texttt{\textbackslash faIcon{camera}} \\
\faCameraHome & \texttt{\textbackslash faCameraHome} & \texttt{\textbackslash faIcon{camera-security}} \\
\faCarAlt & \texttt{\textbackslash faCarAlt} & \texttt{\textbackslash faIcon{car-rear}} \\
\faCarCrash & \texttt{\textbackslash faCarCrash} & \texttt{\textbackslash faIcon{car-burst}} \\
\faCarMechanic & \texttt{\textbackslash faCarMechanic} & \texttt{\textbackslash faIcon{car-wrench}} \\
\faCaravanAlt & \texttt{\textbackslash faCaravanAlt} & \texttt{\textbackslash faIcon{caravan-simple}} \\
\faCaretCircleDown & \texttt{\textbackslash faCaretCircleDown} & \texttt{\textbackslash faIcon{circle-caret-down}} \\
\faCaretCircleLeft & \texttt{\textbackslash faCaretCircleLeft} & \texttt{\textbackslash faIcon{circle-caret-left}} \\
\faCaretCircleRight & \texttt{\textbackslash faCaretCircleRight} & \texttt{\textbackslash faIcon{circle-caret-right}} \\
\faCaretCircleUp & \texttt{\textbackslash faCaretCircleUp} & \texttt{\textbackslash faIcon{circle-caret-up}} \\
\faCaretSquareDown & \texttt{\textbackslash faCaretSquareDown} & \texttt{\textbackslash faIcon{square-caret-down}} \\
\faCaretSquareLeft & \texttt{\textbackslash faCaretSquareLeft} & \texttt{\textbackslash faIcon{square-caret-left}} \\
\faCaretSquareRight & \texttt{\textbackslash faCaretSquareRight} & \texttt{\textbackslash faIcon{square-caret-right}} \\
\faCaretSquareUp & \texttt{\textbackslash faCaretSquareUp} & \texttt{\textbackslash faIcon{square-caret-up}} \\
\faCctv & \texttt{\textbackslash faCctv} & \texttt{\textbackslash faIcon{camera-cctv}} \\
\faChalkboardTeacher & \texttt{\textbackslash faChalkboardTeacher} & \texttt{\textbackslash faIcon{chalkboard-user}} \\
\faChartPieAlt & \texttt{\textbackslash faChartPieAlt} & \texttt{\textbackslash faIcon{chart-pie-simple}} \\
\faCheckCircle & \texttt{\textbackslash faCheckCircle} & \texttt{\textbackslash faIcon{circle-check}} \\
\faCheckSquare & \texttt{\textbackslash faCheckSquare} & \texttt{\textbackslash faIcon{square-check}} \\
\faCheeseburger & \texttt{\textbackslash faCheeseburger} & \texttt{\textbackslash faIcon{burger-cheese}} \\
\faChessBishopAlt & \texttt{\textbackslash faChessBishopAlt} & \texttt{\textbackslash faIcon{chess-bishop-piece}} \\
\faChessClockAlt & \texttt{\textbackslash faChessClockAlt} & \texttt{\textbackslash faIcon{chess-clock-flip}} \\
\faChessKingAlt & \texttt{\textbackslash faChessKingAlt} & \texttt{\textbackslash faIcon{chess-king-piece}} \\
\faChessKnightAlt & \texttt{\textbackslash faChessKnightAlt} & \texttt{\textbackslash faIcon{chess-knight-piece}} \\
\faChessPawnAlt & \texttt{\textbackslash faChessPawnAlt} & \texttt{\textbackslash faIcon{chess-pawn-piece}} \\
\faChessQueenAlt & \texttt{\textbackslash faChessQueenAlt} & \texttt{\textbackslash faIcon{chess-queen-piece}} \\
\faChessRookAlt & \texttt{\textbackslash faChessRookAlt} & \texttt{\textbackslash faIcon{chess-rook-piece}} \\
\faChevronCircleDown & \texttt{\textbackslash faChevronCircleDown} & \texttt{\textbackslash faIcon{circle-chevron-down}} \\
\faChevronCircleLeft & \texttt{\textbackslash faChevronCircleLeft} & \texttt{\textbackslash faIcon{circle-chevron-left}} \\
\faChevronCircleRight & \texttt{\textbackslash faChevronCircleRight} & \texttt{\textbackslash faIcon{circle-chevron-right}} \\
\faChevronCircleUp & \texttt{\textbackslash faChevronCircleUp} & \texttt{\textbackslash faIcon{circle-chevron-up}} \\
\faChevronDoubleDown & \texttt{\textbackslash faChevronDoubleDown} & \texttt{\textbackslash faIcon{chevrons-down}} \\
\faChevronDoubleLeft & \texttt{\textbackslash faChevronDoubleLeft} & \texttt{\textbackslash faIcon{chevrons-left}} \\
\faChevronDoubleRight & \texttt{\textbackslash faChevronDoubleRight} & \texttt{\textbackslash faIcon{chevrons-right}} \\
\faChevronDoubleUp & \texttt{\textbackslash faChevronDoubleUp} & \texttt{\textbackslash faIcon{chevrons-up}} \\
\faChevronSquareDown & \texttt{\textbackslash faChevronSquareDown} & \texttt{\textbackslash faIcon{square-chevron-down}} \\
\faChevronSquareLeft & \texttt{\textbackslash faChevronSquareLeft} & \texttt{\textbackslash faIcon{square-chevron-left}} \\
\faChevronSquareRight & \texttt{\textbackslash faChevronSquareRight} & \texttt{\textbackslash faIcon{square-chevron-right}} \\
\faChevronSquareUp & \texttt{\textbackslash faChevronSquareUp} & \texttt{\textbackslash faIcon{square-chevron-up}} \\
\faClinicMedical & \texttt{\textbackslash faClinicMedical} & \texttt{\textbackslash faIcon{house-chimney-medical}} \\
\faCloudDownload & \texttt{\textbackslash faCloudDownload} & \texttt{\textbackslash faIcon{cloud-arrow-down}} \\
\faCloudDownloadAlt & \texttt{\textbackslash faCloudDownloadAlt} & \texttt{\textbackslash faIcon{cloud-arrow-down}} \\
\faCloudUpload & \texttt{\textbackslash faCloudUpload} & \texttt{\textbackslash faIcon{cloud-arrow-up}} \\
\faCloudUploadAlt & \texttt{\textbackslash faCloudUploadAlt} & \texttt{\textbackslash faIcon{cloud-arrow-up}} \\
\faCocktail & \texttt{\textbackslash faCocktail} & \texttt{\textbackslash faIcon{martini-glass-citrus}} \\
\faCoffee & \texttt{\textbackslash faCoffee} & \texttt{\textbackslash faIcon{mug-saucer}} \\
\faCoffeeTogo & \texttt{\textbackslash faCoffeeTogo} & \texttt{\textbackslash faIcon{cup-togo}} \\
\faCog & \texttt{\textbackslash faCog} & \texttt{\textbackslash faIcon{gear}} \\
\faCogs & \texttt{\textbackslash faCogs} & \texttt{\textbackslash faIcon{gears}} \\
\faColumns & \texttt{\textbackslash faColumns} & \texttt{\textbackslash faIcon{table-columns}} \\
\faCommentAlt & \texttt{\textbackslash faCommentAlt} & \texttt{\textbackslash faIcon{message}} \\
\faCommentAltCheck & \texttt{\textbackslash faCommentAltCheck} & \texttt{\textbackslash faIcon{message-check}} \\
\faCommentAltDollar & \texttt{\textbackslash faCommentAltDollar} & \texttt{\textbackslash faIcon{message-dollar}} \\
\faCommentAltDots & \texttt{\textbackslash faCommentAltDots} & \texttt{\textbackslash faIcon{message-dots}} \\
\faCommentAltEdit & \texttt{\textbackslash faCommentAltEdit} & \texttt{\textbackslash faIcon{message-pen}} \\
\faCommentAltExclamation & \texttt{\textbackslash faCommentAltExclamation} & \texttt{\textbackslash faIcon{message-exclamation}} \\
\faCommentAltLines & \texttt{\textbackslash faCommentAltLines} & \texttt{\textbackslash faIcon{message-lines}} \\
\faCommentAltMedical & \texttt{\textbackslash faCommentAltMedical} & \texttt{\textbackslash faIcon{message-medical}} \\
\faCommentAltMinus & \texttt{\textbackslash faCommentAltMinus} & \texttt{\textbackslash faIcon{message-minus}} \\
\faCommentAltMusic & \texttt{\textbackslash faCommentAltMusic} & \texttt{\textbackslash faIcon{message-music}} \\
\faCommentAltPlus & \texttt{\textbackslash faCommentAltPlus} & \texttt{\textbackslash faIcon{message-plus}} \\
\faCommentAltSlash & \texttt{\textbackslash faCommentAltSlash} & \texttt{\textbackslash faIcon{message-slash}} \\
\faCommentAltSmile & \texttt{\textbackslash faCommentAltSmile} & \texttt{\textbackslash faIcon{message-smile}} \\
\faCommentAltTimes & \texttt{\textbackslash faCommentAltTimes} & \texttt{\textbackslash faIcon{message-xmark}} \\
\faCommentEdit & \texttt{\textbackslash faCommentEdit} & \texttt{\textbackslash faIcon{comment-pen}} \\
\faCommentTimes & \texttt{\textbackslash faCommentTimes} & \texttt{\textbackslash faIcon{comment-xmark}} \\
\faCommentsAlt & \texttt{\textbackslash faCommentsAlt} & \texttt{\textbackslash faIcon{messages}} \\
\faCommentsAltDollar & \texttt{\textbackslash faCommentsAltDollar} & \texttt{\textbackslash faIcon{messages-dollar}} \\
\faCompressAlt & \texttt{\textbackslash faCompressAlt} & \texttt{\textbackslash faIcon{down-left-and-up-right-to-center}} \\
\faCompressArrowsAlt & \texttt{\textbackslash faCompressArrowsAlt} & \texttt{\textbackslash faIcon{minimize}} \\
\faConciergeBell & \texttt{\textbackslash faConciergeBell} & \texttt{\textbackslash faIcon{bell-concierge}} \\
\faConstruction & \texttt{\textbackslash faConstruction} & \texttt{\textbackslash faIcon{triangle-person-digging}} \\
\faConveyorBeltAlt & \texttt{\textbackslash faConveyorBeltAlt} & \texttt{\textbackslash faIcon{conveyor-belt-boxes}} \\
\faCowbellMore & \texttt{\textbackslash faCowbellMore} & \texttt{\textbackslash faIcon{cowbell-circle-plus}} \\
\faCricket & \texttt{\textbackslash faCricket} & \texttt{\textbackslash faIcon{cricket-bat-ball}} \\
\faCropAlt & \texttt{\textbackslash faCropAlt} & \texttt{\textbackslash faIcon{crop-simple}} \\
\faCurling & \texttt{\textbackslash faCurling} & \texttt{\textbackslash faIcon{curling-stone}} \\
\faCut & \texttt{\textbackslash faCut} & \texttt{\textbackslash faIcon{scissors}} \\
\faDeaf & \texttt{\textbackslash faDeaf} & \texttt{\textbackslash faIcon{ear-deaf}} \\
\faDebug & \texttt{\textbackslash faDebug} & \texttt{\textbackslash faIcon{ban-bug}} \\
\faDesktopAlt & \texttt{\textbackslash faDesktopAlt} & \texttt{\textbackslash faIcon{desktop}} \\
\faDewpoint & \texttt{\textbackslash faDewpoint} & \texttt{\textbackslash faIcon{droplet-degree}} \\
\faDiagnoses & \texttt{\textbackslash faDiagnoses} & \texttt{\textbackslash faIcon{person-dots-from-line}} \\
\faDigging & \texttt{\textbackslash faDigging} & \texttt{\textbackslash faIcon{person-digging}} \\
\faDigitalTachograph & \texttt{\textbackslash faDigitalTachograph} & \texttt{\textbackslash faIcon{tachograph-digital}} \\
\faDirections & \texttt{\textbackslash faDirections} & \texttt{\textbackslash faIcon{diamond-turn-right}} \\
\faDizzy & \texttt{\textbackslash faDizzy} & \texttt{\textbackslash faIcon{face-dizzy}} \\
\faDollyFlatbed & \texttt{\textbackslash faDollyFlatbed} & \texttt{\textbackslash faIcon{cart-flatbed}} \\
\faDollyFlatbedAlt & \texttt{\textbackslash faDollyFlatbedAlt} & \texttt{\textbackslash faIcon{cart-flatbed-boxes}} \\
\faDollyFlatbedEmpty & \texttt{\textbackslash faDollyFlatbedEmpty} & \texttt{\textbackslash faIcon{cart-flatbed-empty}} \\
\faDonate & \texttt{\textbackslash faDonate} & \texttt{\textbackslash faIcon{circle-dollar-to-slot}} \\
\faDotCircle & \texttt{\textbackslash faDotCircle} & \texttt{\textbackslash faIcon{circle-dot}} \\
\faDraftingCompass & \texttt{\textbackslash faDraftingCompass} & \texttt{\textbackslash faIcon{compass-drafting}} \\
\faDroneAlt & \texttt{\textbackslash faDroneAlt} & \texttt{\textbackslash faIcon{drone-front}} \\
\faDryerAlt & \texttt{\textbackslash faDryerAlt} & \texttt{\textbackslash faIcon{dryer-heat}} \\
\faEclipseAlt & \texttt{\textbackslash faEclipseAlt} & \texttt{\textbackslash faIcon{moon-over-sun}} \\
\faEdit & \texttt{\textbackslash faEdit} & \texttt{\textbackslash faIcon{pen-to-square}} \\
\faEllipsisH & \texttt{\textbackslash faEllipsisH} & \texttt{\textbackslash faIcon{ellipsis}} \\
\faEllipsisHAlt & \texttt{\textbackslash faEllipsisHAlt} & \texttt{\textbackslash faIcon{ellipsis-stroke}} \\
\faEllipsisV & \texttt{\textbackslash faEllipsisV} & \texttt{\textbackslash faIcon{ellipsis-vertical}} \\
\faEllipsisVAlt & \texttt{\textbackslash faEllipsisVAlt} & \texttt{\textbackslash faIcon{ellipsis-stroke-vertical}} \\
\faEnvelopeSquare & \texttt{\textbackslash faEnvelopeSquare} & \texttt{\textbackslash faIcon{square-envelope}} \\
\faExchange & \texttt{\textbackslash faExchange} & \texttt{\textbackslash faIcon{arrow-right-arrow-left}} \\
\faExchangeAlt & \texttt{\textbackslash faExchangeAlt} & \texttt{\textbackslash faIcon{right-left}} \\
\faExclamationCircle & \texttt{\textbackslash faExclamationCircle} & \texttt{\textbackslash faIcon{circle-exclamation}} \\
\faExclamationSquare & \texttt{\textbackslash faExclamationSquare} & \texttt{\textbackslash faIcon{square-exclamation}} \\
\faExclamationTriangle & \texttt{\textbackslash faExclamationTriangle} & \texttt{\textbackslash faIcon{triangle-exclamation}} \\
\faExpandAlt & \texttt{\textbackslash faExpandAlt} & \texttt{\textbackslash faIcon{up-right-and-down-left-from-center}} \\
\faExpandArrows & \texttt{\textbackslash faExpandArrows} & \texttt{\textbackslash faIcon{arrows-maximize}} \\
\faExpandArrowsAlt & \texttt{\textbackslash faExpandArrowsAlt} & \texttt{\textbackslash faIcon{maximize}} \\
\faExternalLink & \texttt{\textbackslash faExternalLink} & \texttt{\textbackslash faIcon{arrow-up-right-from-square}} \\
\faExternalLinkAlt & \texttt{\textbackslash faExternalLinkAlt} & \texttt{\textbackslash faIcon{up-right-from-square}} \\
\faExternalLinkSquare & \texttt{\textbackslash faExternalLinkSquare} & \texttt{\textbackslash faIcon{square-arrow-up-right}} \\
\faExternalLinkSquareAlt & \texttt{\textbackslash faExternalLinkSquareAlt} & \texttt{\textbackslash faIcon{square-up-right}} \\
\faEyedropper & \texttt{\textbackslash faEyedropper} & \texttt{\textbackslash faIcon{eye-dropper}} \\
\faFastBackward & \texttt{\textbackslash faFastBackward} & \texttt{\textbackslash faIcon{backward-fast}} \\
\faFastForward & \texttt{\textbackslash faFastForward} & \texttt{\textbackslash faIcon{forward-fast}} \\
\faFeatherAlt & \texttt{\textbackslash faFeatherAlt} & \texttt{\textbackslash faIcon{feather-pointed}} \\
\faFemale & \texttt{\textbackslash faFemale} & \texttt{\textbackslash faIcon{person-dress}} \\
\faFieldHockey & \texttt{\textbackslash faFieldHockey} & \texttt{\textbackslash faIcon{field-hockey-stick-ball}} \\
\faFighterJet & \texttt{\textbackslash faFighterJet} & \texttt{\textbackslash faIcon{jet-fighter}} \\
\faFileAlt & \texttt{\textbackslash faFileAlt} & \texttt{\textbackslash faIcon{file-lines}} \\
\faFileArchive & \texttt{\textbackslash faFileArchive} & \texttt{\textbackslash faIcon{file-zipper}} \\
\faFileChartLine & \texttt{\textbackslash faFileChartLine} & \texttt{\textbackslash faIcon{file-chart-column}} \\
\faFileDownload & \texttt{\textbackslash faFileDownload} & \texttt{\textbackslash faIcon{file-arrow-down}} \\
\faFileEdit & \texttt{\textbackslash faFileEdit} & \texttt{\textbackslash faIcon{file-pen}} \\
\faFileMedicalAlt & \texttt{\textbackslash faFileMedicalAlt} & \texttt{\textbackslash faIcon{file-waveform}} \\
\faFileSearch & \texttt{\textbackslash faFileSearch} & \texttt{\textbackslash faIcon{file-magnifying-glass}} \\
\faFileTimes & \texttt{\textbackslash faFileTimes} & \texttt{\textbackslash faIcon{file-xmark}} \\
\faFileUpload & \texttt{\textbackslash faFileUpload} & \texttt{\textbackslash faIcon{file-arrow-up}} \\
\faFilmAlt & \texttt{\textbackslash faFilmAlt} & \texttt{\textbackslash faIcon{film-simple}} \\
\faFireAlt & \texttt{\textbackslash faFireAlt} & \texttt{\textbackslash faIcon{fire-flame-curved}} \\
\faFirstAid & \texttt{\textbackslash faFirstAid} & \texttt{\textbackslash faIcon{kit-medical}} \\
\faFistRaised & \texttt{\textbackslash faFistRaised} & \texttt{\textbackslash faIcon{hand-fist}} \\
\faFlagAlt & \texttt{\textbackslash faFlagAlt} & \texttt{\textbackslash faIcon{flag-swallowtail}} \\
\faFlame & \texttt{\textbackslash faFlame} & \texttt{\textbackslash faIcon{fire-flame}} \\
\faFlaskPoison & \texttt{\textbackslash faFlaskPoison} & \texttt{\textbackslash faIcon{flask-round-poison}} \\
\faFlaskPotion & \texttt{\textbackslash faFlaskPotion} & \texttt{\textbackslash faIcon{flask-round-potion}} \\
\faFlushed & \texttt{\textbackslash faFlushed} & \texttt{\textbackslash faIcon{face-flushed}} \\
\faFog & \texttt{\textbackslash faFog} & \texttt{\textbackslash faIcon{cloud-fog}} \\
\faFolderDownload & \texttt{\textbackslash faFolderDownload} & \texttt{\textbackslash faIcon{folder-arrow-down}} \\
\faFolderTimes & \texttt{\textbackslash faFolderTimes} & \texttt{\textbackslash faIcon{folder-xmark}} \\
\faFolderUpload & \texttt{\textbackslash faFolderUpload} & \texttt{\textbackslash faIcon{folder-arrow-up}} \\
\faFontAwesomeAlt & \texttt{\textbackslash faFontAwesomeAlt} & \texttt{\textbackslash faIcon{square-font-awesome-stroke}} \\
\faFontAwesomeFlag & \texttt{\textbackslash faFontAwesomeFlag} & \texttt{\textbackslash faIcon{font-awesome}} \\
\faFontAwesomeLogoFull & \texttt{\textbackslash faFontAwesomeLogoFull} & \texttt{\textbackslash faIcon{font-awesome}} \\
\faFootballBall & \texttt{\textbackslash faFootballBall} & \texttt{\textbackslash faIcon{football}} \\
\faFragile & \texttt{\textbackslash faFragile} & \texttt{\textbackslash faIcon{wine-glass-crack}} \\
\faFrostyHead & \texttt{\textbackslash faFrostyHead} & \texttt{\textbackslash faIcon{snowman-head}} \\
\faFrown & \texttt{\textbackslash faFrown} & \texttt{\textbackslash faIcon{face-frown}} \\
\faFrownOpen & \texttt{\textbackslash faFrownOpen} & \texttt{\textbackslash faIcon{face-frown-open}} \\
\faFunnelDollar & \texttt{\textbackslash faFunnelDollar} & \texttt{\textbackslash faIcon{filter-circle-dollar}} \\
\faGameBoardAlt & \texttt{\textbackslash faGameBoardAlt} & \texttt{\textbackslash faIcon{game-board-simple}} \\
\faGamepadAlt & \texttt{\textbackslash faGamepadAlt} & \texttt{\textbackslash faIcon{gamepad-modern}} \\
\faGlassChampagne & \texttt{\textbackslash faGlassChampagne} & \texttt{\textbackslash faIcon{champagne-glass}} \\
\faGlassCheers & \texttt{\textbackslash faGlassCheers} & \texttt{\textbackslash faIcon{champagne-glasses}} \\
\faGlassMartini & \texttt{\textbackslash faGlassMartini} & \texttt{\textbackslash faIcon{martini-glass-empty}} \\
\faGlassMartiniAlt & \texttt{\textbackslash faGlassMartiniAlt} & \texttt{\textbackslash faIcon{martini-glass}} \\
\faGlassWhiskey & \texttt{\textbackslash faGlassWhiskey} & \texttt{\textbackslash faIcon{whiskey-glass}} \\
\faGlassWhiskeyRocks & \texttt{\textbackslash faGlassWhiskeyRocks} & \texttt{\textbackslash faIcon{whiskey-glass-ice}} \\
\faGlassesAlt & \texttt{\textbackslash faGlassesAlt} & \texttt{\textbackslash faIcon{glasses-round}} \\
\faGlobeAfrica & \texttt{\textbackslash faGlobeAfrica} & \texttt{\textbackslash faIcon{earth-africa}} \\
\faGlobeAmericas & \texttt{\textbackslash faGlobeAmericas} & \texttt{\textbackslash faIcon{earth-americas}} \\
\faGlobeAsia & \texttt{\textbackslash faGlobeAsia} & \texttt{\textbackslash faIcon{earth-asia}} \\
\faGlobeEurope & \texttt{\textbackslash faGlobeEurope} & \texttt{\textbackslash faIcon{earth-europe}} \\
\faGolfBall & \texttt{\textbackslash faGolfBall} & \texttt{\textbackslash faIcon{golf-ball-tee}} \\
\faGrimace & \texttt{\textbackslash faGrimace} & \texttt{\textbackslash faIcon{face-grimace}} \\
\faGrin & \texttt{\textbackslash faGrin} & \texttt{\textbackslash faIcon{face-grin}} \\
\faGrinAlt & \texttt{\textbackslash faGrinAlt} & \texttt{\textbackslash faIcon{face-grin-wide}} \\
\faGrinBeam & \texttt{\textbackslash faGrinBeam} & \texttt{\textbackslash faIcon{face-grin-beam}} \\
\faGrinBeamSweat & \texttt{\textbackslash faGrinBeamSweat} & \texttt{\textbackslash faIcon{face-grin-beam-sweat}} \\
\faGrinHearts & \texttt{\textbackslash faGrinHearts} & \texttt{\textbackslash faIcon{face-grin-hearts}} \\
\faGrinSquint & \texttt{\textbackslash faGrinSquint} & \texttt{\textbackslash faIcon{face-grin-squint}} \\
\faGrinSquintTears & \texttt{\textbackslash faGrinSquintTears} & \texttt{\textbackslash faIcon{face-grin-squint-tears}} \\
\faGrinStars & \texttt{\textbackslash faGrinStars} & \texttt{\textbackslash faIcon{face-grin-stars}} \\
\faGrinTears & \texttt{\textbackslash faGrinTears} & \texttt{\textbackslash faIcon{face-grin-tears}} \\
\faGrinTongue & \texttt{\textbackslash faGrinTongue} & \texttt{\textbackslash faIcon{face-grin-tongue}} \\
\faGrinTongueSquint & \texttt{\textbackslash faGrinTongueSquint} & \texttt{\textbackslash faIcon{face-grin-tongue-squint}} \\
\faGrinTongueWink & \texttt{\textbackslash faGrinTongueWink} & \texttt{\textbackslash faIcon{face-grin-tongue-wink}} \\
\faGrinWink & \texttt{\textbackslash faGrinWink} & \texttt{\textbackslash faIcon{face-grin-wink}} \\
\faGripHorizontal & \texttt{\textbackslash faGripHorizontal} & \texttt{\textbackslash faIcon{grip}} \\
\faHSquare & \texttt{\textbackslash faHSquare} & \texttt{\textbackslash faIcon{square-h}} \\
\faHamburger & \texttt{\textbackslash faHamburger} & \texttt{\textbackslash faIcon{burger}} \\
\faHandHoldingUsd & \texttt{\textbackslash faHandHoldingUsd} & \texttt{\textbackslash faIcon{hand-holding-dollar}} \\
\faHandHoldingWater & \texttt{\textbackslash faHandHoldingWater} & \texttt{\textbackslash faIcon{hand-holding-droplet}} \\
\faHandPaper & \texttt{\textbackslash faHandPaper} & \texttt{\textbackslash faIcon{hand}} \\
\faHandReceiving & \texttt{\textbackslash faHandReceiving} & \texttt{\textbackslash faIcon{hands-holding-diamond}} \\
\faHandRock & \texttt{\textbackslash faHandRock} & \texttt{\textbackslash faIcon{hand-back-fist}} \\
\faHandsHeart & \texttt{\textbackslash faHandsHeart} & \texttt{\textbackslash faIcon{hands-holding-heart}} \\
\faHandsHelping & \texttt{\textbackslash faHandsHelping} & \texttt{\textbackslash faIcon{handshake-angle}} \\
\faHandsUsd & \texttt{\textbackslash faHandsUsd} & \texttt{\textbackslash faIcon{hands-holding-dollar}} \\
\faHandsWash & \texttt{\textbackslash faHandsWash} & \texttt{\textbackslash faIcon{hands-bubbles}} \\
\faHandshakeAlt & \texttt{\textbackslash faHandshakeAlt} & \texttt{\textbackslash faIcon{handshake-simple}} \\
\faHandshakeAltSlash & \texttt{\textbackslash faHandshakeAltSlash} & \texttt{\textbackslash faIcon{handshake-simple-slash}} \\
\faHardHat & \texttt{\textbackslash faHardHat} & \texttt{\textbackslash faIcon{helmet-safety}} \\
\faHdd & \texttt{\textbackslash faHdd} & \texttt{\textbackslash faIcon{hard-drive}} \\
\faHeadVr & \texttt{\textbackslash faHeadVr} & \texttt{\textbackslash faIcon{head-side-goggles}} \\
\faHeadphonesAlt & \texttt{\textbackslash faHeadphonesAlt} & \texttt{\textbackslash faIcon{headphones-simple}} \\
\faHeartBroken & \texttt{\textbackslash faHeartBroken} & \texttt{\textbackslash faIcon{heart-crack}} \\
\faHeartCircle & \texttt{\textbackslash faHeartCircle} & \texttt{\textbackslash faIcon{circle-heart}} \\
\faHeartRate & \texttt{\textbackslash faHeartRate} & \texttt{\textbackslash faIcon{wave-pulse}} \\
\faHeartSquare & \texttt{\textbackslash faHeartSquare} & \texttt{\textbackslash faIcon{square-heart}} \\
\faHeartbeat & \texttt{\textbackslash faHeartbeat} & \texttt{\textbackslash faIcon{heart-pulse}} \\
\faHiking & \texttt{\textbackslash faHiking} & \texttt{\textbackslash faIcon{person-hiking}} \\
\faHistory & \texttt{\textbackslash faHistory} & \texttt{\textbackslash faIcon{clock-rotate-left}} \\
\faHome & \texttt{\textbackslash faHome} & \texttt{\textbackslash faIcon{house}} \\
\faHomeAlt & \texttt{\textbackslash faHomeAlt} & \texttt{\textbackslash faIcon{house}} \\
\faHomeHeart & \texttt{\textbackslash faHomeHeart} & \texttt{\textbackslash faIcon{house-heart}} \\
\faHomeLg & \texttt{\textbackslash faHomeLg} & \texttt{\textbackslash faIcon{house-chimney}} \\
\faHomeLgAlt & \texttt{\textbackslash faHomeLgAlt} & \texttt{\textbackslash faIcon{house}} \\
\faHospitalAlt & \texttt{\textbackslash faHospitalAlt} & \texttt{\textbackslash faIcon{hospital}} \\
\faHospitalSymbol & \texttt{\textbackslash faHospitalSymbol} & \texttt{\textbackslash faIcon{circle-h}} \\
\faHotTub & \texttt{\textbackslash faHotTub} & \texttt{\textbackslash faIcon{hot-tub-person}} \\
\faHouseDamage & \texttt{\textbackslash faHouseDamage} & \texttt{\textbackslash faIcon{house-chimney-crack}} \\
\faHouseLeave & \texttt{\textbackslash faHouseLeave} & \texttt{\textbackslash faIcon{house-person-leave}} \\
\faHouseReturn & \texttt{\textbackslash faHouseReturn} & \texttt{\textbackslash faIcon{house-person-return}} \\
\faHryvnia & \texttt{\textbackslash faHryvnia} & \texttt{\textbackslash faIcon{hryvnia-sign}} \\
\faHumidity & \texttt{\textbackslash faHumidity} & \texttt{\textbackslash faIcon{droplet-percent}} \\
\faIconsAlt & \texttt{\textbackslash faIconsAlt} & \texttt{\textbackslash faIcon{symbols}} \\
\faIdCardAlt & \texttt{\textbackslash faIdCardAlt} & \texttt{\textbackslash faIcon{id-card-clip}} \\
\faIndustryAlt & \texttt{\textbackslash faIndustryAlt} & \texttt{\textbackslash faIcon{industry-windows}} \\
\faInfoCircle & \texttt{\textbackslash faInfoCircle} & \texttt{\textbackslash faIcon{circle-info}} \\
\faInfoSquare & \texttt{\textbackslash faInfoSquare} & \texttt{\textbackslash faIcon{square-info}} \\
\faInnosoft & \texttt{\textbackslash faInnosoft} & \texttt{\textbackslash faIcon{42-group}} \\
\faInventory & \texttt{\textbackslash faInventory} & \texttt{\textbackslash faIcon{shelves}} \\
\faJournalWhills & \texttt{\textbackslash faJournalWhills} & \texttt{\textbackslash faIcon{book-journal-whills}} \\
\faKiss & \texttt{\textbackslash faKiss} & \texttt{\textbackslash faIcon{face-kiss}} \\
\faKissBeam & \texttt{\textbackslash faKissBeam} & \texttt{\textbackslash faIcon{face-kiss-beam}} \\
\faKissWinkHeart & \texttt{\textbackslash faKissWinkHeart} & \texttt{\textbackslash faIcon{face-kiss-wink-heart}} \\
\faLandmarkAlt & \texttt{\textbackslash faLandmarkAlt} & \texttt{\textbackslash faIcon{landmark-dome}} \\
\faLaptopHouse & \texttt{\textbackslash faLaptopHouse} & \texttt{\textbackslash faIcon{house-laptop}} \\
\faLaugh & \texttt{\textbackslash faLaugh} & \texttt{\textbackslash faIcon{face-laugh}} \\
\faLaughBeam & \texttt{\textbackslash faLaughBeam} & \texttt{\textbackslash faIcon{face-laugh-beam}} \\
\faLaughSquint & \texttt{\textbackslash faLaughSquint} & \texttt{\textbackslash faIcon{face-laugh-squint}} \\
\faLaughWink & \texttt{\textbackslash faLaughWink} & \texttt{\textbackslash faIcon{face-laugh-wink}} \\
\faLevelDown & \texttt{\textbackslash faLevelDown} & \texttt{\textbackslash faIcon{arrow-turn-down}} \\
\faLevelDownAlt & \texttt{\textbackslash faLevelDownAlt} & \texttt{\textbackslash faIcon{turn-down}} \\
\faLevelUp & \texttt{\textbackslash faLevelUp} & \texttt{\textbackslash faIcon{arrow-turn-up}} \\
\faLevelUpAlt & \texttt{\textbackslash faLevelUpAlt} & \texttt{\textbackslash faIcon{turn-up}} \\
\faListAlt & \texttt{\textbackslash faListAlt} & \texttt{\textbackslash faIcon{rectangle-list}} \\
\faLocation & \texttt{\textbackslash faLocation} & \texttt{\textbackslash faIcon{location-crosshairs}} \\
\faLocationCircle & \texttt{\textbackslash faLocationCircle} & \texttt{\textbackslash faIcon{circle-location-arrow}} \\
\faLocationSlash & \texttt{\textbackslash faLocationSlash} & \texttt{\textbackslash faIcon{location-crosshairs-slash}} \\
\faLockAlt & \texttt{\textbackslash faLockAlt} & \texttt{\textbackslash faIcon{lock-keyhole}} \\
\faLockOpenAlt & \texttt{\textbackslash faLockOpenAlt} & \texttt{\textbackslash faIcon{lock-keyhole-open}} \\
\faLongArrowAltDown & \texttt{\textbackslash faLongArrowAltDown} & \texttt{\textbackslash faIcon{down-long}} \\
\faLongArrowAltLeft & \texttt{\textbackslash faLongArrowAltLeft} & \texttt{\textbackslash faIcon{left-long}} \\
\faLongArrowAltRight & \texttt{\textbackslash faLongArrowAltRight} & \texttt{\textbackslash faIcon{right-long}} \\
\faLongArrowAltUp & \texttt{\textbackslash faLongArrowAltUp} & \texttt{\textbackslash faIcon{up-long}} \\
\faLongArrowDown & \texttt{\textbackslash faLongArrowDown} & \texttt{\textbackslash faIcon{arrow-down-long}} \\
\faLongArrowLeft & \texttt{\textbackslash faLongArrowLeft} & \texttt{\textbackslash faIcon{arrow-left-long}} \\
\faLongArrowRight & \texttt{\textbackslash faLongArrowRight} & \texttt{\textbackslash faIcon{arrow-right-long}} \\
\faLongArrowUp & \texttt{\textbackslash faLongArrowUp} & \texttt{\textbackslash faIcon{arrow-up-long}} \\
\faLowVision & \texttt{\textbackslash faLowVision} & \texttt{\textbackslash faIcon{eye-low-vision}} \\
\faLuchador & \texttt{\textbackslash faLuchador} & \texttt{\textbackslash faIcon{luchador-mask}} \\
\faLuggageCart & \texttt{\textbackslash faLuggageCart} & \texttt{\textbackslash faIcon{cart-flatbed-suitcase}} \\
\faMagic & \texttt{\textbackslash faMagic} & \texttt{\textbackslash faIcon{wand-magic}} \\
\faMailBulk & \texttt{\textbackslash faMailBulk} & \texttt{\textbackslash faIcon{envelopes-bulk}} \\
\faMale & \texttt{\textbackslash faMale} & \texttt{\textbackslash faIcon{person}} \\
\faMapMarked & \texttt{\textbackslash faMapMarked} & \texttt{\textbackslash faIcon{map-location}} \\
\faMapMarkedAlt & \texttt{\textbackslash faMapMarkedAlt} & \texttt{\textbackslash faIcon{map-location-dot}} \\
\faMapMarker & \texttt{\textbackslash faMapMarker} & \texttt{\textbackslash faIcon{location-pin}} \\
\faMapMarkerAlt & \texttt{\textbackslash faMapMarkerAlt} & \texttt{\textbackslash faIcon{location-dot}} \\
\faMapMarkerAltSlash & \texttt{\textbackslash faMapMarkerAltSlash} & \texttt{\textbackslash faIcon{location-dot-slash}} \\
\faMapMarkerCheck & \texttt{\textbackslash faMapMarkerCheck} & \texttt{\textbackslash faIcon{location-check}} \\
\faMapMarkerEdit & \texttt{\textbackslash faMapMarkerEdit} & \texttt{\textbackslash faIcon{location-pen}} \\
\faMapMarkerExclamation & \texttt{\textbackslash faMapMarkerExclamation} & \texttt{\textbackslash faIcon{location-exclamation}} \\
\faMapMarkerMinus & \texttt{\textbackslash faMapMarkerMinus} & \texttt{\textbackslash faIcon{location-minus}} \\
\faMapMarkerPlus & \texttt{\textbackslash faMapMarkerPlus} & \texttt{\textbackslash faIcon{location-plus}} \\
\faMapMarkerQuestion & \texttt{\textbackslash faMapMarkerQuestion} & \texttt{\textbackslash faIcon{location-question}} \\
\faMapMarkerSlash & \texttt{\textbackslash faMapMarkerSlash} & \texttt{\textbackslash faIcon{location-pin-slash}} \\
\faMapMarkerSmile & \texttt{\textbackslash faMapMarkerSmile} & \texttt{\textbackslash faIcon{location-smile}} \\
\faMapMarkerTimes & \texttt{\textbackslash faMapMarkerTimes} & \texttt{\textbackslash faIcon{location-xmark}} \\
\faMapSigns & \texttt{\textbackslash faMapSigns} & \texttt{\textbackslash faIcon{signs-post}} \\
\faMarsStrokeH & \texttt{\textbackslash faMarsStrokeH} & \texttt{\textbackslash faIcon{mars-stroke-right}} \\
\faMarsStrokeV & \texttt{\textbackslash faMarsStrokeV} & \texttt{\textbackslash faIcon{mars-stroke-up}} \\
\faMediumM & \texttt{\textbackslash faMediumM} & \texttt{\textbackslash faIcon{medium}} \\
\faMedkit & \texttt{\textbackslash faMedkit} & \texttt{\textbackslash faIcon{suitcase-medical}} \\
\faMeh & \texttt{\textbackslash faMeh} & \texttt{\textbackslash faIcon{face-meh}} \\
\faMehBlank & \texttt{\textbackslash faMehBlank} & \texttt{\textbackslash faIcon{face-meh-blank}} \\
\faMehRollingEyes & \texttt{\textbackslash faMehRollingEyes} & \texttt{\textbackslash faIcon{face-rolling-eyes}} \\
\faMicrophoneAlt & \texttt{\textbackslash faMicrophoneAlt} & \texttt{\textbackslash faIcon{microphone-lines}} \\
\faMicrophoneAltSlash & \texttt{\textbackslash faMicrophoneAltSlash} & \texttt{\textbackslash faIcon{microphone-lines-slash}} \\
\faMindShare & \texttt{\textbackslash faMindShare} & \texttt{\textbackslash faIcon{brain-arrow-curved-right}} \\
\faMinusCircle & \texttt{\textbackslash faMinusCircle} & \texttt{\textbackslash faIcon{circle-minus}} \\
\faMinusHexagon & \texttt{\textbackslash faMinusHexagon} & \texttt{\textbackslash faIcon{hexagon-minus}} \\
\faMinusOctagon & \texttt{\textbackslash faMinusOctagon} & \texttt{\textbackslash faIcon{octagon-minus}} \\
\faMinusSquare & \texttt{\textbackslash faMinusSquare} & \texttt{\textbackslash faIcon{square-minus}} \\
\faMobileAlt & \texttt{\textbackslash faMobileAlt} & \texttt{\textbackslash faIcon{mobile-screen-button}} \\
\faMobileAndroid & \texttt{\textbackslash faMobileAndroid} & \texttt{\textbackslash faIcon{mobile}} \\
\faMobileAndroidAlt & \texttt{\textbackslash faMobileAndroidAlt} & \texttt{\textbackslash faIcon{mobile-screen}} \\
\faMoneyBillAlt & \texttt{\textbackslash faMoneyBillAlt} & \texttt{\textbackslash faIcon{money-bill-1}} \\
\faMoneyBillWaveAlt & \texttt{\textbackslash faMoneyBillWaveAlt} & \texttt{\textbackslash faIcon{money-bill-1-wave}} \\
\faMoneyCheckAlt & \texttt{\textbackslash faMoneyCheckAlt} & \texttt{\textbackslash faIcon{money-check-dollar}} \\
\faMoneyCheckEdit & \texttt{\textbackslash faMoneyCheckEdit} & \texttt{\textbackslash faIcon{money-check-pen}} \\
\faMoneyCheckEditAlt & \texttt{\textbackslash faMoneyCheckEditAlt} & \texttt{\textbackslash faIcon{money-check-dollar-pen}} \\
\faMonitorHeartRate & \texttt{\textbackslash faMonitorHeartRate} & \texttt{\textbackslash faIcon{monitor-waveform}} \\
\faMouse & \texttt{\textbackslash faMouse} & \texttt{\textbackslash faIcon{computer-mouse}} \\
\faMouseAlt & \texttt{\textbackslash faMouseAlt} & \texttt{\textbackslash faIcon{computer-mouse-scrollwheel}} \\
\faMousePointer & \texttt{\textbackslash faMousePointer} & \texttt{\textbackslash faIcon{arrow-pointer}} \\
\faMusicAlt & \texttt{\textbackslash faMusicAlt} & \texttt{\textbackslash faIcon{music-note}} \\
\faMusicAltSlash & \texttt{\textbackslash faMusicAltSlash} & \texttt{\textbackslash faIcon{music-note-slash}} \\
\faOilTemp & \texttt{\textbackslash faOilTemp} & \texttt{\textbackslash faIcon{oil-temperature}} \\
\faPageBreak & \texttt{\textbackslash faPageBreak} & \texttt{\textbackslash faIcon{file-dashed-line}} \\
\faPaintBrush & \texttt{\textbackslash faPaintBrush} & \texttt{\textbackslash faIcon{paintbrush}} \\
\faPaintBrushAlt & \texttt{\textbackslash faPaintBrushAlt} & \texttt{\textbackslash faIcon{paintbrush-fine}} \\
\faPaintBrushFine & \texttt{\textbackslash faPaintBrushFine} & \texttt{\textbackslash faIcon{paintbrush-fine}} \\
\faPalletAlt & \texttt{\textbackslash faPalletAlt} & \texttt{\textbackslash faIcon{pallet-boxes}} \\
\faParagraphRtl & \texttt{\textbackslash faParagraphRtl} & \texttt{\textbackslash faIcon{paragraph-left}} \\
\faParking & \texttt{\textbackslash faParking} & \texttt{\textbackslash faIcon{square-parking}} \\
\faParkingCircle & \texttt{\textbackslash faParkingCircle} & \texttt{\textbackslash faIcon{circle-parking}} \\
\faParkingCircleSlash & \texttt{\textbackslash faParkingCircleSlash} & \texttt{\textbackslash faIcon{ban-parking}} \\
\faParkingSlash & \texttt{\textbackslash faParkingSlash} & \texttt{\textbackslash faIcon{square-parking-slash}} \\
\faPastafarianism & \texttt{\textbackslash faPastafarianism} & \texttt{\textbackslash faIcon{spaghetti-monster-flying}} \\
\faPauseCircle & \texttt{\textbackslash faPauseCircle} & \texttt{\textbackslash faIcon{circle-pause}} \\
\faPawAlt & \texttt{\textbackslash faPawAlt} & \texttt{\textbackslash faIcon{paw-simple}} \\
\faPenAlt & \texttt{\textbackslash faPenAlt} & \texttt{\textbackslash faIcon{pen-clip}} \\
\faPenSquare & \texttt{\textbackslash faPenSquare} & \texttt{\textbackslash faIcon{square-pen}} \\
\faPencilAlt & \texttt{\textbackslash faPencilAlt} & \texttt{\textbackslash faIcon{pencil}} \\
\faPencilPaintbrush & \texttt{\textbackslash faPencilPaintbrush} & \texttt{\textbackslash faIcon{pen-paintbrush}} \\
\faPencilRuler & \texttt{\textbackslash faPencilRuler} & \texttt{\textbackslash faIcon{pen-ruler}} \\
\faPennant & \texttt{\textbackslash faPennant} & \texttt{\textbackslash faIcon{flag-pennant}} \\
\faPeopleCarry & \texttt{\textbackslash faPeopleCarry} & \texttt{\textbackslash faIcon{people-carry-box}} \\
\faPercentage & \texttt{\textbackslash faPercentage} & \texttt{\textbackslash faIcon{percent}} \\
\faPersonCarry & \texttt{\textbackslash faPersonCarry} & \texttt{\textbackslash faIcon{person-carry-box}} \\
\faPhoneAlt & \texttt{\textbackslash faPhoneAlt} & \texttt{\textbackslash faIcon{phone-flip}} \\
\faPhoneLaptop & \texttt{\textbackslash faPhoneLaptop} & \texttt{\textbackslash faIcon{laptop-mobile}} \\
\faPhoneSquare & \texttt{\textbackslash faPhoneSquare} & \texttt{\textbackslash faIcon{square-phone}} \\
\faPhoneSquareAlt & \texttt{\textbackslash faPhoneSquareAlt} & \texttt{\textbackslash faIcon{square-phone-flip}} \\
\faPhotoVideo & \texttt{\textbackslash faPhotoVideo} & \texttt{\textbackslash faIcon{photo-film}} \\
\faPlaneAlt & \texttt{\textbackslash faPlaneAlt} & \texttt{\textbackslash faIcon{plane-engines}} \\
\faPlayCircle & \texttt{\textbackslash faPlayCircle} & \texttt{\textbackslash faIcon{circle-play}} \\
\faPlusCircle & \texttt{\textbackslash faPlusCircle} & \texttt{\textbackslash faIcon{circle-plus}} \\
\faPlusHexagon & \texttt{\textbackslash faPlusHexagon} & \texttt{\textbackslash faIcon{hexagon-plus}} \\
\faPlusOctagon & \texttt{\textbackslash faPlusOctagon} & \texttt{\textbackslash faIcon{octagon-plus}} \\
\faPlusSquare & \texttt{\textbackslash faPlusSquare} & \texttt{\textbackslash faIcon{square-plus}} \\
\faPoll & \texttt{\textbackslash faPoll} & \texttt{\textbackslash faIcon{square-poll-vertical}} \\
\faPollH & \texttt{\textbackslash faPollH} & \texttt{\textbackslash faIcon{square-poll-horizontal}} \\
\faPortalEnter & \texttt{\textbackslash faPortalEnter} & \texttt{\textbackslash faIcon{person-to-portal}} \\
\faPortalExit & \texttt{\textbackslash faPortalExit} & \texttt{\textbackslash faIcon{person-from-portal}} \\
\faPortrait & \texttt{\textbackslash faPortrait} & \texttt{\textbackslash faIcon{image-portrait}} \\
\faPoundSign & \texttt{\textbackslash faPoundSign} & \texttt{\textbackslash faIcon{sterling-sign}} \\
\faPray & \texttt{\textbackslash faPray} & \texttt{\textbackslash faIcon{person-praying}} \\
\faPrayingHands & \texttt{\textbackslash faPrayingHands} & \texttt{\textbackslash faIcon{hands-praying}} \\
\faPrescriptionBottleAlt & \texttt{\textbackslash faPrescriptionBottleAlt} & \texttt{\textbackslash faIcon{prescription-bottle-medical}} \\
\faPresentation & \texttt{\textbackslash faPresentation} & \texttt{\textbackslash faIcon{presentation-screen}} \\
\faPrintSearch & \texttt{\textbackslash faPrintSearch} & \texttt{\textbackslash faIcon{print-magnifying-glass}} \\
\faProcedures & \texttt{\textbackslash faProcedures} & \texttt{\textbackslash faIcon{bed-pulse}} \\
\faProjectDiagram & \texttt{\textbackslash faProjectDiagram} & \texttt{\textbackslash faIcon{diagram-project}} \\
\faQuestionCircle & \texttt{\textbackslash faQuestionCircle} & \texttt{\textbackslash faIcon{circle-question}} \\
\faQuestionSquare & \texttt{\textbackslash faQuestionSquare} & \texttt{\textbackslash faIcon{square-question}} \\
\faQuran & \texttt{\textbackslash faQuran} & \texttt{\textbackslash faIcon{book-quran}} \\
\faRabbitFast & \texttt{\textbackslash faRabbitFast} & \texttt{\textbackslash faIcon{rabbit-running}} \\
\faRadiationAlt & \texttt{\textbackslash faRadiationAlt} & \texttt{\textbackslash faIcon{circle-radiation}} \\
\faRadioAlt & \texttt{\textbackslash faRadioAlt} & \texttt{\textbackslash faIcon{radio-tuner}} \\
\faRandom & \texttt{\textbackslash faRandom} & \texttt{\textbackslash faIcon{shuffle}} \\
\faRectangleLandscape & \texttt{\textbackslash faRectangleLandscape} & \texttt{\textbackslash faIcon{rectangle}} \\
\faRectanglePortrait & \texttt{\textbackslash faRectanglePortrait} & \texttt{\textbackslash faIcon{rectangle-vertical}} \\
\faRedo & \texttt{\textbackslash faRedo} & \texttt{\textbackslash faIcon{arrow-rotate-right}} \\
\faRedoAlt & \texttt{\textbackslash faRedoAlt} & \texttt{\textbackslash faIcon{rotate-right}} \\
\faRemoveFormat & \texttt{\textbackslash faRemoveFormat} & \texttt{\textbackslash faIcon{text-slash}} \\
\faRepeat1Alt & \texttt{\textbackslash faRepeat1Alt} & \texttt{\textbackslash faIcon{arrows-repeat-1}} \\
\faRepeatAlt & \texttt{\textbackslash faRepeatAlt} & \texttt{\textbackslash faIcon{arrows-repeat}} \\
\faRetweetAlt & \texttt{\textbackslash faRetweetAlt} & \texttt{\textbackslash faIcon{arrows-retweet}} \\
\faRssSquare & \texttt{\textbackslash faRssSquare} & \texttt{\textbackslash faIcon{square-rss}} \\
\faRunning & \texttt{\textbackslash faRunning} & \texttt{\textbackslash faIcon{person-running}} \\
\faSadCry & \texttt{\textbackslash faSadCry} & \texttt{\textbackslash faIcon{face-sad-cry}} \\
\faSadTear & \texttt{\textbackslash faSadTear} & \texttt{\textbackslash faIcon{face-sad-tear}} \\
\faSave & \texttt{\textbackslash faSave} & \texttt{\textbackslash faIcon{floppy-disk}} \\
\faSaxHot & \texttt{\textbackslash faSaxHot} & \texttt{\textbackslash faIcon{saxophone-fire}} \\
\faScalpelPath & \texttt{\textbackslash faScalpelPath} & \texttt{\textbackslash faIcon{scalpel-line-dashed}} \\
\faScannerImage & \texttt{\textbackslash faScannerImage} & \texttt{\textbackslash faIcon{scanner}} \\
\faSearch & \texttt{\textbackslash faSearch} & \texttt{\textbackslash faIcon{magnifying-glass}} \\
\faSearchDollar & \texttt{\textbackslash faSearchDollar} & \texttt{\textbackslash faIcon{magnifying-glass-dollar}} \\
\faSearchLocation & \texttt{\textbackslash faSearchLocation} & \texttt{\textbackslash faIcon{magnifying-glass-location}} \\
\faSearchMinus & \texttt{\textbackslash faSearchMinus} & \texttt{\textbackslash faIcon{magnifying-glass-minus}} \\
\faSearchPlus & \texttt{\textbackslash faSearchPlus} & \texttt{\textbackslash faIcon{magnifying-glass-plus}} \\
\faSensorAlert & \texttt{\textbackslash faSensorAlert} & \texttt{\textbackslash faIcon{sensor-triangle-exclamation}} \\
\faSensorSmoke & \texttt{\textbackslash faSensorSmoke} & \texttt{\textbackslash faIcon{sensor-cloud}} \\
\faShareAlt & \texttt{\textbackslash faShareAlt} & \texttt{\textbackslash faIcon{share-nodes}} \\
\faShareAltSquare & \texttt{\textbackslash faShareAltSquare} & \texttt{\textbackslash faIcon{square-share-nodes}} \\
\faShareSquare & \texttt{\textbackslash faShareSquare} & \texttt{\textbackslash faIcon{share-from-square}} \\
\faShieldAlt & \texttt{\textbackslash faShieldAlt} & \texttt{\textbackslash faIcon{shield-blank}} \\
\faShippingFast & \texttt{\textbackslash faShippingFast} & \texttt{\textbackslash faIcon{truck-fast}} \\
\faShippingTimed & \texttt{\textbackslash faShippingTimed} & \texttt{\textbackslash faIcon{truck-clock}} \\
\faShoppingBag & \texttt{\textbackslash faShoppingBag} & \texttt{\textbackslash faIcon{bag-shopping}} \\
\faShoppingBasket & \texttt{\textbackslash faShoppingBasket} & \texttt{\textbackslash faIcon{basket-shopping}} \\
\faShoppingCart & \texttt{\textbackslash faShoppingCart} & \texttt{\textbackslash faIcon{cart-shopping}} \\
\faShuttleVan & \texttt{\textbackslash faShuttleVan} & \texttt{\textbackslash faIcon{van-shuttle}} \\
\faSign & \texttt{\textbackslash faSign} & \texttt{\textbackslash faIcon{sign-hanging}} \\
\faSignIn & \texttt{\textbackslash faSignIn} & \texttt{\textbackslash faIcon{arrow-right-to-bracket}} \\
\faSignInAlt & \texttt{\textbackslash faSignInAlt} & \texttt{\textbackslash faIcon{right-to-bracket}} \\
\faSignLanguage & \texttt{\textbackslash faSignLanguage} & \texttt{\textbackslash faIcon{hands}} \\
\faSignOut & \texttt{\textbackslash faSignOut} & \texttt{\textbackslash faIcon{arrow-right-from-bracket}} \\
\faSignOutAlt & \texttt{\textbackslash faSignOutAlt} & \texttt{\textbackslash faIcon{right-from-bracket}} \\
\faSignal1 & \texttt{\textbackslash faSignal1} & \texttt{\textbackslash faIcon{signal-weak}} \\
\faSignal2 & \texttt{\textbackslash faSignal2} & \texttt{\textbackslash faIcon{signal-fair}} \\
\faSignal3 & \texttt{\textbackslash faSignal3} & \texttt{\textbackslash faIcon{signal-good}} \\
\faSignal4 & \texttt{\textbackslash faSignal4} & \texttt{\textbackslash faIcon{signal-strong}} \\
\faSignalAlt & \texttt{\textbackslash faSignalAlt} & \texttt{\textbackslash faIcon{signal-bars}} \\
\faSignalAlt1 & \texttt{\textbackslash faSignalAlt1} & \texttt{\textbackslash faIcon{signal-bars-weak}} \\
\faSignalAlt2 & \texttt{\textbackslash faSignalAlt2} & \texttt{\textbackslash faIcon{signal-bars-fair}} \\
\faSignalAlt3 & \texttt{\textbackslash faSignalAlt3} & \texttt{\textbackslash faIcon{signal-bars-good}} \\
\faSignalAltSlash & \texttt{\textbackslash faSignalAltSlash} & \texttt{\textbackslash faIcon{signal-bars-slash}} \\
\faSkating & \texttt{\textbackslash faSkating} & \texttt{\textbackslash faIcon{person-skating}} \\
\faSkiJump & \texttt{\textbackslash faSkiJump} & \texttt{\textbackslash faIcon{person-ski-jumping}} \\
\faSkiLift & \texttt{\textbackslash faSkiLift} & \texttt{\textbackslash faIcon{person-ski-lift}} \\
\faSkiing & \texttt{\textbackslash faSkiing} & \texttt{\textbackslash faIcon{person-skiing}} \\
\faSkiingNordic & \texttt{\textbackslash faSkiingNordic} & \texttt{\textbackslash faIcon{person-skiing-nordic}} \\
\faSlackHash & \texttt{\textbackslash faSlackHash} & \texttt{\textbackslash faIcon{slack}} \\
\faSledding & \texttt{\textbackslash faSledding} & \texttt{\textbackslash faIcon{person-sledding}} \\
\faSlidersH & \texttt{\textbackslash faSlidersH} & \texttt{\textbackslash faIcon{sliders}} \\
\faSlidersHSquare & \texttt{\textbackslash faSlidersHSquare} & \texttt{\textbackslash faIcon{square-sliders}} \\
\faSlidersV & \texttt{\textbackslash faSlidersV} & \texttt{\textbackslash faIcon{sliders-up}} \\
\faSlidersVSquare & \texttt{\textbackslash faSlidersVSquare} & \texttt{\textbackslash faIcon{square-sliders-vertical}} \\
\faSmile & \texttt{\textbackslash faSmile} & \texttt{\textbackslash faIcon{face-smile}} \\
\faSmileBeam & \texttt{\textbackslash faSmileBeam} & \texttt{\textbackslash faIcon{face-smile-beam}} \\
\faSmilePlus & \texttt{\textbackslash faSmilePlus} & \texttt{\textbackslash faIcon{face-smile-plus}} \\
\faSmileWink & \texttt{\textbackslash faSmileWink} & \texttt{\textbackslash faIcon{face-smile-wink}} \\
\faSmokingBan & \texttt{\textbackslash faSmokingBan} & \texttt{\textbackslash faIcon{ban-smoking}} \\
\faSms & \texttt{\textbackslash faSms} & \texttt{\textbackslash faIcon{comment-sms}} \\
\faSnapchatGhost & \texttt{\textbackslash faSnapchatGhost} & \texttt{\textbackslash faIcon{snapchat}} \\
\faSnowboarding & \texttt{\textbackslash faSnowboarding} & \texttt{\textbackslash faIcon{person-snowboarding}} \\
\faSnowmobile & \texttt{\textbackslash faSnowmobile} & \texttt{\textbackslash faIcon{person-snowmobiling}} \\
\faSortAlphaDown & \texttt{\textbackslash faSortAlphaDown} & \texttt{\textbackslash faIcon{arrow-down-a-z}} \\
\faSortAlphaDownAlt & \texttt{\textbackslash faSortAlphaDownAlt} & \texttt{\textbackslash faIcon{arrow-down-z-a}} \\
\faSortAlphaUp & \texttt{\textbackslash faSortAlphaUp} & \texttt{\textbackslash faIcon{arrow-up-a-z}} \\
\faSortAlphaUpAlt & \texttt{\textbackslash faSortAlphaUpAlt} & \texttt{\textbackslash faIcon{arrow-up-z-a}} \\
\faSortAlt & \texttt{\textbackslash faSortAlt} & \texttt{\textbackslash faIcon{arrow-down-arrow-up}} \\
\faSortAmountDown & \texttt{\textbackslash faSortAmountDown} & \texttt{\textbackslash faIcon{arrow-down-wide-short}} \\
\faSortAmountDownAlt & \texttt{\textbackslash faSortAmountDownAlt} & \texttt{\textbackslash faIcon{arrow-down-short-wide}} \\
\faSortAmountUp & \texttt{\textbackslash faSortAmountUp} & \texttt{\textbackslash faIcon{arrow-up-wide-short}} \\
\faSortAmountUpAlt & \texttt{\textbackslash faSortAmountUpAlt} & \texttt{\textbackslash faIcon{arrow-up-short-wide}} \\
\faSortCircle & \texttt{\textbackslash faSortCircle} & \texttt{\textbackslash faIcon{circle-sort}} \\
\faSortCircleDown & \texttt{\textbackslash faSortCircleDown} & \texttt{\textbackslash faIcon{circle-sort-down}} \\
\faSortCircleUp & \texttt{\textbackslash faSortCircleUp} & \texttt{\textbackslash faIcon{circle-sort-up}} \\
\faSortNumericDown & \texttt{\textbackslash faSortNumericDown} & \texttt{\textbackslash faIcon{arrow-down-1-9}} \\
\faSortNumericDownAlt & \texttt{\textbackslash faSortNumericDownAlt} & \texttt{\textbackslash faIcon{arrow-down-9-1}} \\
\faSortNumericUp & \texttt{\textbackslash faSortNumericUp} & \texttt{\textbackslash faIcon{arrow-up-1-9}} \\
\faSortNumericUpAlt & \texttt{\textbackslash faSortNumericUpAlt} & \texttt{\textbackslash faIcon{arrow-up-9-1}} \\
\faSortShapesDown & \texttt{\textbackslash faSortShapesDown} & \texttt{\textbackslash faIcon{arrow-down-triangle-square}} \\
\faSortShapesDownAlt & \texttt{\textbackslash faSortShapesDownAlt} & \texttt{\textbackslash faIcon{arrow-down-square-triangle}} \\
\faSortShapesUp & \texttt{\textbackslash faSortShapesUp} & \texttt{\textbackslash faIcon{arrow-up-triangle-square}} \\
\faSortShapesUpAlt & \texttt{\textbackslash faSortShapesUpAlt} & \texttt{\textbackslash faIcon{arrow-up-square-triangle}} \\
\faSortSizeDown & \texttt{\textbackslash faSortSizeDown} & \texttt{\textbackslash faIcon{arrow-down-big-small}} \\
\faSortSizeDownAlt & \texttt{\textbackslash faSortSizeDownAlt} & \texttt{\textbackslash faIcon{arrow-down-small-big}} \\
\faSortSizeUp & \texttt{\textbackslash faSortSizeUp} & \texttt{\textbackslash faIcon{arrow-up-big-small}} \\
\faSortSizeUpAlt & \texttt{\textbackslash faSortSizeUpAlt} & \texttt{\textbackslash faIcon{arrow-up-small-big}} \\
\faSoup & \texttt{\textbackslash faSoup} & \texttt{\textbackslash faIcon{bowl-hot}} \\
\faSpaceShuttle & \texttt{\textbackslash faSpaceShuttle} & \texttt{\textbackslash faIcon{shuttle-space}} \\
\faSpaceStationMoonAlt & \texttt{\textbackslash faSpaceStationMoonAlt} & \texttt{\textbackslash faIcon{space-station-moon-construction}} \\
\faSquareRootAlt & \texttt{\textbackslash faSquareRootAlt} & \texttt{\textbackslash faIcon{square-root-variable}} \\
\faStarHalfAlt & \texttt{\textbackslash faStarHalfAlt} & \texttt{\textbackslash faIcon{star-half-stroke}} \\
\faStarfighterAlt & \texttt{\textbackslash faStarfighterAlt} & \texttt{\textbackslash faIcon{starfighter-twin-ion-engine}} \\
\faStepBackward & \texttt{\textbackslash faStepBackward} & \texttt{\textbackslash faIcon{backward-step}} \\
\faStepForward & \texttt{\textbackslash faStepForward} & \texttt{\textbackslash faIcon{forward-step}} \\
\faStickyNote & \texttt{\textbackslash faStickyNote} & \texttt{\textbackslash faIcon{note-sticky}} \\
\faStopCircle & \texttt{\textbackslash faStopCircle} & \texttt{\textbackslash faIcon{circle-stop}} \\
\faStoreAlt & \texttt{\textbackslash faStoreAlt} & \texttt{\textbackslash faIcon{shop}} \\
\faStoreAltSlash & \texttt{\textbackslash faStoreAltSlash} & \texttt{\textbackslash faIcon{shop-slash}} \\
\faStream & \texttt{\textbackslash faStream} & \texttt{\textbackslash faIcon{bars-staggered}} \\
\faSubway & \texttt{\textbackslash faSubway} & \texttt{\textbackslash faIcon{train-subway}} \\
\faSurprise & \texttt{\textbackslash faSurprise} & \texttt{\textbackslash faIcon{face-surprise}} \\
\faSwimmer & \texttt{\textbackslash faSwimmer} & \texttt{\textbackslash faIcon{person-swimming}} \\
\faSwimmingPool & \texttt{\textbackslash faSwimmingPool} & \texttt{\textbackslash faIcon{water-ladder}} \\
\faSync & \texttt{\textbackslash faSync} & \texttt{\textbackslash faIcon{arrows-rotate}} \\
\faSyncAlt & \texttt{\textbackslash faSyncAlt} & \texttt{\textbackslash faIcon{rotate}} \\
\faTableTennis & \texttt{\textbackslash faTableTennis} & \texttt{\textbackslash faIcon{table-tennis-paddle-ball}} \\
\faTabletAlt & \texttt{\textbackslash faTabletAlt} & \texttt{\textbackslash faIcon{tablet-screen-button}} \\
\faTabletAndroid & \texttt{\textbackslash faTabletAndroid} & \texttt{\textbackslash faIcon{tablet}} \\
\faTabletAndroidAlt & \texttt{\textbackslash faTabletAndroidAlt} & \texttt{\textbackslash faIcon{tablet-screen}} \\
\faTachometer & \texttt{\textbackslash faTachometer} & \texttt{\textbackslash faIcon{gauge-simple}} \\
\faTachometerAlt & \texttt{\textbackslash faTachometerAlt} & \texttt{\textbackslash faIcon{gauge}} \\
\faTachometerAltAverage & \texttt{\textbackslash faTachometerAltAverage} & \texttt{\textbackslash faIcon{gauge-med}} \\
\faTachometerAltFast & \texttt{\textbackslash faTachometerAltFast} & \texttt{\textbackslash faIcon{gauge}} \\
\faTachometerAltFastest & \texttt{\textbackslash faTachometerAltFastest} & \texttt{\textbackslash faIcon{gauge-max}} \\
\faTachometerAltSlow & \texttt{\textbackslash faTachometerAltSlow} & \texttt{\textbackslash faIcon{gauge-low}} \\
\faTachometerAltSlowest & \texttt{\textbackslash faTachometerAltSlowest} & \texttt{\textbackslash faIcon{gauge-min}} \\
\faTachometerAverage & \texttt{\textbackslash faTachometerAverage} & \texttt{\textbackslash faIcon{gauge-simple-med}} \\
\faTachometerFast & \texttt{\textbackslash faTachometerFast} & \texttt{\textbackslash faIcon{gauge-simple}} \\
\faTachometerFastest & \texttt{\textbackslash faTachometerFastest} & \texttt{\textbackslash faIcon{gauge-simple-max}} \\
\faTachometerSlow & \texttt{\textbackslash faTachometerSlow} & \texttt{\textbackslash faIcon{gauge-simple-low}} \\
\faTachometerSlowest & \texttt{\textbackslash faTachometerSlowest} & \texttt{\textbackslash faIcon{gauge-simple-min}} \\
\faTanakh & \texttt{\textbackslash faTanakh} & \texttt{\textbackslash faIcon{book-tanakh}} \\
\faTasks & \texttt{\textbackslash faTasks} & \texttt{\textbackslash faIcon{list-check}} \\
\faTasksAlt & \texttt{\textbackslash faTasksAlt} & \texttt{\textbackslash faIcon{bars-progress}} \\
\faTelegramPlane & \texttt{\textbackslash faTelegramPlane} & \texttt{\textbackslash faIcon{telegram}} \\
\faTemperatureDown & \texttt{\textbackslash faTemperatureDown} & \texttt{\textbackslash faIcon{temperature-arrow-down}} \\
\faTemperatureFrigid & \texttt{\textbackslash faTemperatureFrigid} & \texttt{\textbackslash faIcon{temperature-snow}} \\
\faTemperatureHot & \texttt{\textbackslash faTemperatureHot} & \texttt{\textbackslash faIcon{temperature-sun}} \\
\faTemperatureUp & \texttt{\textbackslash faTemperatureUp} & \texttt{\textbackslash faIcon{temperature-arrow-up}} \\
\faTenge & \texttt{\textbackslash faTenge} & \texttt{\textbackslash faIcon{tenge-sign}} \\
\faTh & \texttt{\textbackslash faTh} & \texttt{\textbackslash faIcon{table-cells}} \\
\faThLarge & \texttt{\textbackslash faThLarge} & \texttt{\textbackslash faIcon{table-cells-large}} \\
\faThList & \texttt{\textbackslash faThList} & \texttt{\textbackslash faIcon{table-list}} \\
\faTheaterMasks & \texttt{\textbackslash faTheaterMasks} & \texttt{\textbackslash faIcon{masks-theater}} \\
\faThermometerEmpty & \texttt{\textbackslash faThermometerEmpty} & \texttt{\textbackslash faIcon{temperature-empty}} \\
\faThermometerFull & \texttt{\textbackslash faThermometerFull} & \texttt{\textbackslash faIcon{temperature-full}} \\
\faThermometerHalf & \texttt{\textbackslash faThermometerHalf} & \texttt{\textbackslash faIcon{temperature-half}} \\
\faThermometerQuarter & \texttt{\textbackslash faThermometerQuarter} & \texttt{\textbackslash faIcon{temperature-quarter}} \\
\faThermometerThreeQuarters & \texttt{\textbackslash faThermometerThreeQuarters} & \texttt{\textbackslash faIcon{temperature-three-quarters}} \\
\faThumbTack & \texttt{\textbackslash faThumbTack} & \texttt{\textbackslash faIcon{thumbtack}} \\
\faThunderstorm & \texttt{\textbackslash faThunderstorm} & \texttt{\textbackslash faIcon{cloud-bolt}} \\
\faThunderstormMoon & \texttt{\textbackslash faThunderstormMoon} & \texttt{\textbackslash faIcon{cloud-bolt-moon}} \\
\faThunderstormSun & \texttt{\textbackslash faThunderstormSun} & \texttt{\textbackslash faIcon{cloud-bolt-sun}} \\
\faTicketAlt & \texttt{\textbackslash faTicketAlt} & \texttt{\textbackslash faIcon{ticket-simple}} \\
\faTimes & \texttt{\textbackslash faTimes} & \texttt{\textbackslash faIcon{xmark}} \\
\faTimesCircle & \texttt{\textbackslash faTimesCircle} & \texttt{\textbackslash faIcon{circle-xmark}} \\
\faTimesHexagon & \texttt{\textbackslash faTimesHexagon} & \texttt{\textbackslash faIcon{hexagon-xmark}} \\
\faTimesOctagon & \texttt{\textbackslash faTimesOctagon} & \texttt{\textbackslash faIcon{octagon-xmark}} \\
\faTimesSquare & \texttt{\textbackslash faTimesSquare} & \texttt{\textbackslash faIcon{square-xmark}} \\
\faTint & \texttt{\textbackslash faTint} & \texttt{\textbackslash faIcon{droplet}} \\
\faTintSlash & \texttt{\textbackslash faTintSlash} & \texttt{\textbackslash faIcon{droplet-slash}} \\
\faTired & \texttt{\textbackslash faTired} & \texttt{\textbackslash faIcon{face-tired}} \\
\faToiletPaperAlt & \texttt{\textbackslash faToiletPaperAlt} & \texttt{\textbackslash faIcon{toilet-paper-blank}} \\
\faTombstoneAlt & \texttt{\textbackslash faTombstoneAlt} & \texttt{\textbackslash faIcon{tombstone-blank}} \\
\faTools & \texttt{\textbackslash faTools} & \texttt{\textbackslash faIcon{screwdriver-wrench}} \\
\faTorah & \texttt{\textbackslash faTorah} & \texttt{\textbackslash faIcon{scroll-torah}} \\
\faTram & \texttt{\textbackslash faTram} & \texttt{\textbackslash faIcon{train-tram}} \\
\faTransgenderAlt & \texttt{\textbackslash faTransgenderAlt} & \texttt{\textbackslash faIcon{transgender}} \\
\faTrashAlt & \texttt{\textbackslash faTrashAlt} & \texttt{\textbackslash faIcon{trash-can}} \\
\faTrashRestore & \texttt{\textbackslash faTrashRestore} & \texttt{\textbackslash faIcon{trash-arrow-up}} \\
\faTrashRestoreAlt & \texttt{\textbackslash faTrashRestoreAlt} & \texttt{\textbackslash faIcon{trash-can-arrow-up}} \\
\faTrashUndoAlt & \texttt{\textbackslash faTrashUndoAlt} & \texttt{\textbackslash faIcon{trash-can-undo}} \\
\faTreeAlt & \texttt{\textbackslash faTreeAlt} & \texttt{\textbackslash faIcon{tree-deciduous}} \\
\faTriangleMusic & \texttt{\textbackslash faTriangleMusic} & \texttt{\textbackslash faIcon{triangle-instrument}} \\
\faTrophyAlt & \texttt{\textbackslash faTrophyAlt} & \texttt{\textbackslash faIcon{trophy-star}} \\
\faTruckCouch & \texttt{\textbackslash faTruckCouch} & \texttt{\textbackslash faIcon{truck-ramp-couch}} \\
\faTruckLoading & \texttt{\textbackslash faTruckLoading} & \texttt{\textbackslash faIcon{truck-ramp-box}} \\
\faTshirt & \texttt{\textbackslash faTshirt} & \texttt{\textbackslash faIcon{shirt}} \\
\faTvAlt & \texttt{\textbackslash faTvAlt} & \texttt{\textbackslash faIcon{tv}} \\
\faUndo & \texttt{\textbackslash faUndo} & \texttt{\textbackslash faIcon{arrow-rotate-left}} \\
\faUndoAlt & \texttt{\textbackslash faUndoAlt} & \texttt{\textbackslash faIcon{rotate-left}} \\
\faUniversity & \texttt{\textbackslash faUniversity} & \texttt{\textbackslash faIcon{building-columns}} \\
\faUnlink & \texttt{\textbackslash faUnlink} & \texttt{\textbackslash faIcon{link-slash}} \\
\faUnlockAlt & \texttt{\textbackslash faUnlockAlt} & \texttt{\textbackslash faIcon{unlock-keyhole}} \\
\faUsdCircle & \texttt{\textbackslash faUsdCircle} & \texttt{\textbackslash faIcon{circle-dollar}} \\
\faUsdSquare & \texttt{\textbackslash faUsdSquare} & \texttt{\textbackslash faIcon{square-dollar}} \\
\faUserAlt & \texttt{\textbackslash faUserAlt} & \texttt{\textbackslash faIcon{user-large}} \\
\faUserAltSlash & \texttt{\textbackslash faUserAltSlash} & \texttt{\textbackslash faIcon{user-large-slash}} \\
\faUserChart & \texttt{\textbackslash faUserChart} & \texttt{\textbackslash faIcon{chart-user}} \\
\faUserCircle & \texttt{\textbackslash faUserCircle} & \texttt{\textbackslash faIcon{circle-user}} \\
\faUserCog & \texttt{\textbackslash faUserCog} & \texttt{\textbackslash faIcon{user-gear}} \\
\faUserEdit & \texttt{\textbackslash faUserEdit} & \texttt{\textbackslash faIcon{user-pen}} \\
\faUserFriends & \texttt{\textbackslash faUserFriends} & \texttt{\textbackslash faIcon{user-group}} \\
\faUserHardHat & \texttt{\textbackslash faUserHardHat} & \texttt{\textbackslash faIcon{user-helmet-safety}} \\
\faUserMd & \texttt{\textbackslash faUserMd} & \texttt{\textbackslash faIcon{user-doctor}} \\
\faUserMdChat & \texttt{\textbackslash faUserMdChat} & \texttt{\textbackslash faIcon{user-doctor-message}} \\
\faUserTimes & \texttt{\textbackslash faUserTimes} & \texttt{\textbackslash faIcon{user-xmark}} \\
\faUsersClass & \texttt{\textbackslash faUsersClass} & \texttt{\textbackslash faIcon{screen-users}} \\
\faUsersCog & \texttt{\textbackslash faUsersCog} & \texttt{\textbackslash faIcon{users-gear}} \\
\faUsersCrown & \texttt{\textbackslash faUsersCrown} & \texttt{\textbackslash faIcon{user-group-crown}} \\
\faUtensilFork & \texttt{\textbackslash faUtensilFork} & \texttt{\textbackslash faIcon{fork}} \\
\faUtensilKnife & \texttt{\textbackslash faUtensilKnife} & \texttt{\textbackslash faIcon{knife}} \\
\faUtensilSpoon & \texttt{\textbackslash faUtensilSpoon} & \texttt{\textbackslash faIcon{spoon}} \\
\faUtensilsAlt & \texttt{\textbackslash faUtensilsAlt} & \texttt{\textbackslash faIcon{fork-knife}} \\
\faVhs & \texttt{\textbackslash faVhs} & \texttt{\textbackslash faIcon{cassette-vhs}} \\
\faVolleyballBall & \texttt{\textbackslash faVolleyballBall} & \texttt{\textbackslash faIcon{volleyball}} \\
\faVolumeDown & \texttt{\textbackslash faVolumeDown} & \texttt{\textbackslash faIcon{volume-low}} \\
\faVolumeMute & \texttt{\textbackslash faVolumeMute} & \texttt{\textbackslash faIcon{volume-xmark}} \\
\faVolumeUp & \texttt{\textbackslash faVolumeUp} & \texttt{\textbackslash faIcon{volume-high}} \\
\faVoteNay & \texttt{\textbackslash faVoteNay} & \texttt{\textbackslash faIcon{xmark-to-slot}} \\
\faVoteYea & \texttt{\textbackslash faVoteYea} & \texttt{\textbackslash faIcon{check-to-slot}} \\
\faWalking & \texttt{\textbackslash faWalking} & \texttt{\textbackslash faIcon{person-walking}} \\
\faWarehouseAlt & \texttt{\textbackslash faWarehouseAlt} & \texttt{\textbackslash faIcon{warehouse-full}} \\
\faWasher & \texttt{\textbackslash faWasher} & \texttt{\textbackslash faIcon{washing-machine}} \\
\faWaterLower & \texttt{\textbackslash faWaterLower} & \texttt{\textbackslash faIcon{water-arrow-down}} \\
\faWaterRise & \texttt{\textbackslash faWaterRise} & \texttt{\textbackslash faIcon{water-arrow-up}} \\
\faWaveformPath & \texttt{\textbackslash faWaveformPath} & \texttt{\textbackslash faIcon{waveform-lines}} \\
\faWebcam & \texttt{\textbackslash faWebcam} & \texttt{\textbackslash faIcon{camera-web}} \\
\faWebcamSlash & \texttt{\textbackslash faWebcamSlash} & \texttt{\textbackslash faIcon{camera-web-slash}} \\
\faWeight & \texttt{\textbackslash faWeight} & \texttt{\textbackslash faIcon{weight-scale}} \\
\faWifi1 & \texttt{\textbackslash faWifi1} & \texttt{\textbackslash faIcon{wifi-weak}} \\
\faWifi2 & \texttt{\textbackslash faWifi2} & \texttt{\textbackslash faIcon{wifi-fair}} \\
\faWindowAlt & \texttt{\textbackslash faWindowAlt} & \texttt{\textbackslash faIcon{window-flip}} \\
\faWindowClose & \texttt{\textbackslash faWindowClose} & \texttt{\textbackslash faIcon{rectangle-xmark}} \\
\faWineGlassAlt & \texttt{\textbackslash faWineGlassAlt} & \texttt{\textbackslash faIcon{wine-glass-empty}} \\
\end{longtable}
\restoregeometry

\end{document}
